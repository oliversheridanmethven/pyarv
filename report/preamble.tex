% Author: 
%
%	Oliver Sheridan-Methven.
%	
% Description:
%
%	A collection of extremely useful packages 
%	which have been accumulated over a long time,
%	all of which combine to make very nice LaTeX
%	documents. 



\usepackage{adjustbox} % Nice alternative to minipage.
\usepackage{afterpage} % To give the title page its own geometry.
\usepackage[boxed,noline,linesnumbered,resetcount,algosection,noend]{algorithm2e}
\usepackage{algpseudocode}

\usepackage{amsmath} % Nice maths symbols.
\usepackage{amssymb} % Nice variable symbols.
\let\amsDiamond=\Diamond
\usepackage{amsthm}
\usepackage{array} % Allow for custom column widths in tables.
\usepackage{ltablex} % For long tables spanning multiple pages. % Must be before ARYDSHLN package!
\keepXColumns % Keeps the X column
\usepackage{arydshln} % Dashed lines using \hdashline \cdashline
\usepackage[english]{babel} % (I only use this to give blindmath math symbols.)
\usepackage{bbm} % Gives Blackboard fonts.
\usepackage{blindtext} % Generates dummy maths. cf. lipsum.
\usepackage{calc} % Calculates widths of words. 
\let\counterwithout\relax
\let\counterwithin\relax
\usepackage{chngcntr} % Changing counters, e.g. with footnotes.
\usepackage{cuted} % For single column modes in twocolumn mode. 
%\AtBeginEnvironment{strip}{\leavevmode}
%\AfterEndEnvironment{strip}{\leavevmode} % cf. https://tex.stackexchange.com/a/264551/106804
\usepackage{cprotect} % For verbatim like environments in todonotes. 
\usepackage{datenumber}
\usepackage{datetime}
\usepackage[scale=3, color={[gray]{0.9}}]{draftwatermark} % Gives a draft overlay. Use options [nostamp] or [final].
\usepackage[spaced=-1,mleftright]{diffcoeff}
\usepackage{emptypage} % Empty pages have no headers and footers.
\usepackage{paralist} % Must be loaded before enumitem.
\usepackage{enumitem} % Nice listing options in itemize and enumerate. 
%\usepackage{enumerate} % A seemingly nicer package than enumitem. PREFER ENUMITEM cf. https://tex.stackexchange.com/a/519982/106804
\usepackage{etoolbox} % For defining conditionals. 
\usepackage{fancyhdr} % Nice headers.
\usepackage{float} % Nice figure placement.
\usepackage[T1]{fontenc} % Nice range of text characters and accents.
\usepackage[bottom,multiple]{footmisc} % Nice footnote formatting.
\usepackage{graphicx} % Include figures.
\usepackage[notquote]{hanging} % For indenting later lines in a paragraph. USE the 'noquote' option else the `'` is overwritten, and breaks in maths mode!

% Sadly using Arial font is a constraint, and Helvet is the closest thing that LaTeX can easily offer. 
%\renewcommand{\familydefault}{\sfdefault}
%\usepackage[scaled=1]{helvet}
%\usepackage[helvet]{sfmath}
%\everymath={\sf}
%\renewcommand{\ttdefault}{pcr}


\usepackage{ifoddpage} % Checks for odd or even page.
\usepackage[geometry]{ifsym} % Useful symbols.
\usepackage{imakeidx} % Makes the index.
\usepackage{indentfirst} % Indents the first paragraph.
\usepackage{letltxmacro} % For defining a nice SQRT symbol.
\usepackage[switch, columnwise]{lineno} % For line numbers. 
\usepackage{lipsum} % Useful for adding jargon.
\usepackage{listings} % The listings package for code.
\usepackage{marginnote} % For nice margin notes.
\usepackage{mathtools} % Gives the colon equals symbol.
\usepackage[framed,numbered,useliterate]{mcode} % Inports Listings package ideal for MATLAB.
\usepackage[framemethod=tikz]{mdframed} % Gives nices boxed and sidesrules.
\usepackage{mleftright}
\usepackage{multirow} % Nice table cells spanning many rows.
\usepackage{multicol} % If I want to use multiple columns.
\usepackage[numbers, sort&compress]{natbib} % Nice references.
%\usepackage{sansmath} % Gives a changing math font. %% Must be before newpxtext/newpxmath ! %%
%\usepackage{newpxtext} % Gives Palatino and Helvetica fonts.
%\usepackage{newpxmath} % Gives Palatino and Helvetica fonts in maths. 
\usepackage{bm} % Bold math symbols. %% Needs to be loaded after newpxtext/math. %%
\usepackage{nicefrac} % Gives nice fractions for superscripts.
\usepackage{nomencl} % Gives a symbol nomenclature. 
\usepackage[super]{nth} % Gives nice ordinal superscripts, eg 1st, 2nd, etc.
\usepackage{parskip} % Gives nicer indenting.
\usepackage{ragged2e} % For nice allignment.
\usepackage{realboxes}
%\usepackage[norefs]{refcheck} % Can show any unused references.
\usepackage{romannum} % Nice typing for roman numerals.
\usepackage{rotating} % For sideways figures.
\usepackage[scr,scaled=1.1]{rsfso} % Gives Script fonts which are not so slanted. 
%\usepackage[subtle]{savetrees}  % Choose from options: [subtle, moderate, extreme]
\usepackage{setspace} % Ideal for increasing line spacing. E.g.  \doublespacing
%\usepackage{showframe}
\usepackage{siunitx} % Nice formating of units.
\usepackage{soul}
%\usepackage{sidenotes} % Nice margin figures and margin tables. % Doesn't play well with figure* in twocolumn mode, and should be commented out in such cases. 
\usepackage{subcaption} % Side by side figures.
\usepackage[textsize=small]{todonotes} % A nice TODO list. [disable] to supress.
\usepackage{tikz} % Nice diagrams.
\usepackage{titling} % Access title variables. 
\usepackage{titlesec} % Nice section title colouring options.
\usepackage[nottoc]{tocbibind} % Gives nices Table of Contents
\usepackage[hyphens,spaces,obeyspaces]{url}
\usepackage{verbatimbox} % For verbatim inside todonotes. 
\usepackage{wasysym}
\usepackage{xcolor} % This is useful for making greyed table cells, nice for headers. Known preamble placement issues.
\usepackage{xifthen}% Provides \isempty test.
\usepackage{xparse} % Gives \NewDocumentEnvironment which has nice optional argument handling.
\usepackage{xpatch}
\usepackage{xspace} % Gives nice spacing for commands.
\usepackage{xurl} % Nicer leanbreaks for URLs. 
%%%% Generally HYPERREF should be imported last. %%%%
\usepackage[colorlinks=true,linkcolor=black,urlcolor=black,citecolor=black,anchorcolor=black]{hyperref} % Colour links.
%%%% Should be loaded after hyperref. %%%%
\usepackage[noabbrev]{cleveref} % Gives smart referencing. %% After Hyperref
\usepackage[margin=10pt,font=small,labelfont=bf,labelsep=endash,figurewithin=section,tablewithin=section]{caption} % Caption figures and tables nicely. %% After cleveref.
\usepackage[top=10mm, right=20mm, left=20mm, bottom=60mm]{geometry} % Use nice margins. Does give a small change in the default page margins. 
\usepackage{graphicx}

% Spacing between columns in multicolumn settings. 
%\setlength{\columnsep}{10mm}

% Ensures subsecs are numbered.
\setcounter{secnumdepth}{3}

% Making an index. 
\makeindex

% Making a nomenclature. 
\makenomenclature
% The column width for any nomenclature. 
\setlength\nomlabelwidth{0.2\linewidth}

% Set the table of content depth to only subsections. 
\setcounter{tocdepth}{2}

% Makes math bold in headers and titles. 
\makeatletter
\g@addto@macro\bfseries{\boldmath}
\makeatother

% Fixes line breaks before subeqtuations. (cf. https://tex.stackexchange.com/a/27058/106804)
\preto\subequations{\ifhmode\unskip\fi}


% Nicer algorithms
\SetAlCapSkip{1em} % Increase algorithm spacing between box and caption.
\SetAlCapNameFnt{\small}
\SetAlCapFnt{\small}
\SetAlFnt{\small}
\SetAlgoCaptionSeparator{ \textendash}
\setlength{\algomargin}{2em}
\makeatletter
 %Remove right hand margin in algorithm
\patchcmd{\@algocf@start}% <cmd>
  {-1.5em}% <search>
  {-1em}% <replace>
{}{}% <success><failure>
\makeatother
\newcommand{\algrule}[1][0.2pt]{\par\vskip.2\baselineskip\hrule height #1\par\vskip.2\baselineskip}
%cf. https://tex.stackexchange.com/a/323331/106804
% remove all the "do, then" statements
\SetKw{KwGoTo}{go to}
\SetKwIF{If}{ElseIf}{Else}{if}{}{else if}{else}{end if}%
\SetKwFor{While}{while}{}{endw}%
\SetKwFor{ForEach}{for each}{}{endfch}
\SetKwRepeat{Do}{do}{while}
\SetInd{1em}{0em}


% Have theorems and lemmas be referenced by name if applicable.
\makeatletter
\renewrobustcmd{\cref}{\@osmcref{cref}}
\renewrobustcmd{\Cref}{\@osmcref{Cref}}
\def\@osmcref#1#2{%
    \begingroup
    \ifcsundef{r@#2}
    {}
    {\expandafter\expandafter\expandafter\expandafter\expandafter
        \expandafter\expandafter\def
        \expandafter\expandafter\expandafter\expandafter\expandafter
        \expandafter\expandafter\@osmcref@name
        \expandafter\expandafter\expandafter\expandafter\expandafter
        \expandafter\expandafter{%
            \expandafter\expandafter\expandafter
            \@thirdoffive\csname r@#2\endcsname}}%
    \ifcsundef{r@#2@cref}
    {}
    {\cref@gettype{#2}{\@osmcref@type}}%
    \ifboolexpr{not test {\ifdefvoid{\@osmcref@name}}
        and (test {\ifdefstring{\@osmcref@type}{theorem}}
        or test {\ifdefstring{\@osmcref@type}{lemma}}
        or test {\ifdefstring{\@osmcref@type}{corollary}})}
    {\nameref{#2} (\@cref{#1}{#2})}
    {\@cref{#1}{#2}}%
    \endgroup
}
\makeatother

% Giving correct theorem and lemma environments. 
\newtheorem{theorem}{Theorem}[section]
\newtheorem{corollary}{Corollary}[theorem]
\newtheorem{proposition}[theorem]{Proposition}
\newtheorem{lemma}[theorem]{Lemma}
\theoremstyle{definition}
\newtheorem{remark}{Remark}[section]
\newtheorem{definition}{Definition}[section]
\newtheorem{assumption}{Assumption}[section]
% Naming these nicely.
\crefname{lemma}{lemma}{lemmas}
\Crefname{lemma}{Lemma}{Lemmas}
\crefname{theorem}{theorem}{theorems}
\Crefname{theorem}{Theorem}{Theorems}
\crefname{corollary}{corollary}{corollaries}
\Crefname{corollary}{Corollary}{Corollaries}
\Crefname{proposition}{proposition}{propositions}
\Crefname{proposition}{Proposition}{Propositions}
\crefname{remark}{remark}{remarks}
\Crefname{remark}{Remark}{Remarks}
\crefname{definition}{definition}{definitions}
\Crefname{definition}{Definition}{Definitions}
\crefname{assumption}{assumption}{assumptions}
\Crefname{assumption}{Assumption}{Assumptions}
% Change the end of proof symbol
\renewcommand\qedsymbol{\textbf{QED}}
% Making the proof in bold so it stands out more.  (cf. https://tex.stackexchange.com/a/724697/106804)
\newcommand{\prooffont}{\upshape\bfseries}
\xpatchcmd{\proof}{\itshape}{\prooffont}{}{}

% Give bold names to definitions and similar environments. 
\makeatletter
\def\th@plain{%
    \thm@notefont{}% same as heading font
    \itshape % body font
}
\def\th@definition{%
    \thm@notefont{}% same as heading font
    \normalfont % body font
}
\makeatother


% Supressing bad box warnings
%\hbadness=10000 

% Where to search for figures. 
\graphicspath{{.}{../figures/}}

% Present the references in the order they are used.
%\bibliographystyle{unsrtnat}

\bibliographystyle{plainnat}
% Reduce spacing between references. 
\setlength{\bibsep}{0pt plus 0.3ex}

% Listing -> Code in environment labels.
\renewcommand{\lstlistingname}{Code}
\crefname{listing}{code}{codes}
\Crefname{listing}{Code}{Codes}
\definecolor{codegray}{rgb}{0.9,0.9,0.9}
\lstset{
    numbers=left, 
    basicstyle=\ttfamily\footnotesize,
    frame=single, % adds a frame around the code
    xleftmargin=20.4pt,
    xrightmargin=3.4pt,
    %	numbersep=3mm,
    aboveskip=0.5em,
    belowskip=0.em,
    breaklines=true,
    breakatwhitespace=true,
    backgroundcolor=\color{codegray},
}
\newfloat{lstfloat}{htbp}{lop} % environment for placing lisings in to make them float. 

% Nice paragraph indents.
\setlength{\parindent}{2em}

% Giving the references the right title.
\renewcommand{\bibname}{References}
\renewcommand{\bibsection}{\section*{References}} % For Natbib
\renewcommand{\refname}{References}
\addto\captionsenglish{\renewcommand\bibname{References}} % When using babel
\addto\captionsenglish{\renewcommand\refname{References}} % When using babel
\renewcommand{\listfigurename}{List of figures}
\renewcommand{\listtablename}{List of tables}

% Removes most hyphenation.
\tolerance=1
\emergencystretch=\maxdimen
\hyphenpenalty=10000
\hbadness=10000

% To change the spacing in lists:
%\setlist{noitemsep} % or \setlist{noitemsep} to leave space around whole list
%% or
%\setenumerate{itemsep=-0.2em,topsep=0.5em} % Seems to look nice.

% Custom column widths using C{2cm}, L, R, etc.
\newcolumntype{L}[1]{>{\raggedright\let\newline\\\arraybackslash\hspace{0pt}}m{#1}}
\newcolumntype{C}[1]{>{\centering\let\newline\\\arraybackslash\hspace{0pt}}m{#1}}
\newcolumntype{R}[1]{>{\raggedleft\let\newline\\\arraybackslash\hspace{0pt}}m{#1}}

% Gives a nice column separation in multicolumn mode.
\setlength{\columnsep}{5mm}

% Figure environment for use in multicolumn. To put in captions use \captionof{figure}{content of caption}.
\newenvironment{Figure}
{\par\medskip\noindent\minipage{\linewidth}}
{\endminipage\par\medskip}

% Gives the nice SQRT symbol.
\makeatletter
\let\oldr@@t\r@@t
\def\r@@t#1#2{%
    \setbox0=\hbox{\(\oldr@@t#1{#2\,}\)}\dimen0=\ht0
    \advance\dimen0-0.2\ht0
    \setbox2=\hbox{\vrule height\ht0 depth -\dimen0}%
    {\box0\lower0.4pt\box2}}
\LetLtxMacro{\oldsqrt}{\sqrt}
\renewcommand*{\sqrt}[2][\ ]{\oldsqrt[#1]{#2}}
\makeatother

% Some common math operators which need their own typesetting.
\DeclareMathOperator{\sign}{sign}
\DeclareMathOperator*{\argmin}{argmin}
\DeclareMathOperator*{\argmax}{argmax}

% Number equations down to the subection level, e.g. 1.2.3 is the third equation in
% subsection 2 of section 1.
\numberwithin{equation}{section}
\newcommand*\tageq{\refstepcounter{equation}\tag{\theequation}}
% Ensure equations are correctly formatted with cleverref
\crefname{equation}{}{}
\creflabelformat{equation}{(#2#1#3)}

% Nice spacing in the first row of a table
\newcommand{\firstrowspacing}{\rule{0pt}{2.6ex}}
% For a more open look in tables.
\setlength\extrarowheight{3pt} 

% For nice headers and footers.
\pagestyle{fancy}
% Specifying the headers and footers. 
\fancyhf{}
\renewcommand*{\footrulewidth}{1pt}%
\renewcommand*{\headrulewidth}{0pt}%
\fancyfoot[L]{Notes:}
\fancypagestyle{plain}{%
	\fancyhf{}%
	\renewcommand*{\headrulewidth}{0pt}%
}
\cfoot{\raisebox{-45mm}{\thepage}}

% Nice spacing in lists
\setlist{listparindent=\parindent,parsep=1ex} 

% The oxford comma from cref for multiple citations. 
\newcommand{\creflastconjunction}{, and\nobreakspace}

% Define the dummy sentence, an ancient palindrome.
\def\sator{Sator Arepo tenet opera rotas.\xspace}

% A command to print the sentence repeatedly.
% Argument #1 is the number of times to repeat it.
\newcount\loopcounter
\def\dummysentences#1{%
    \loopcounter = #1
    \loop
    \sator\ %
    \advance\loopcounter by -1
    \ifnum\loopcounter > 0
    \repeat%
}


% Enable blind maths. 
\blindmathtrue

% Ensures the SI command detects the local environment (e.g. in captions).
\sisetup{detect-weight=true, detect-family=true, 
     group-separator = {\text{'}},
     group-minimum-digits=4
     }

\NewCommandCopy\oldtodo\todo
\outer\def\todo[#1]{\icprotect{\oldtodo[#1]}}


\makeatletter
\xpretocmd\lstinline{\Colorbox{codegray}\bgroup\appto\lst@DeInit{\egroup}}{}{}
\makeatother

\lstdefinestyle{longline}
{numbers=none,
frame=none, 
xleftmargin=0.2em, 
xrightmargin=0em, 
framexleftmargin=0.2em,
}


\newcommand{\mylistlabelfont}[1]{\normalfont\textit{#1:}}
\newlist{longdescription}{description}{1}
\setlist[longdescription]{%
  style=unboxed,
  font=\mylistlabelfont
}


\newcommand{\inlinec}{\lstinline[language=C,basicstyle=\ttfamily]}
\newcommand{\inlineplain}{\lstinline[language={},basicstyle=\ttfamily]}

%\newcommand*{\lstitem}[1]{
%  \setbox0\hbox{\lstinline[language={},basicstyle=\bfseries\ttfamily]{#1}}  
%  \item[\usebox0] 
%}

%to show the frame in two column mode with showfame package cf. https://tex.stackexchange.com/a/271294/106804
%\newlength\Fcolumnseprule
%\setlength\Fcolumnseprule{0.4pt}
%
%\makeatletter
%\def\@outputdblcol{%
%  \if@firstcolumn
%    \global \@firstcolumnfalse
%    \global \setbox\@leftcolumn \box\@outputbox
%  \else
%    \global \@firstcolumntrue
%    \setbox\@outputbox \vbox {%
%                         \hb@xt@\textwidth {%
%                           \hb@xt@\columnwidth {%
%                             \box\@leftcolumn \hss}%
%                           \vrule \@width\Fcolumnseprule\hfil
%                                {\normalcolor\vrule \@width\columnseprule}%original:
%                                                %\normalcolor\vrule \@width\columnseprule
%                           \hfil\vrule \@width\Fcolumnseprule
%                           \hb@xt@\columnwidth {%
%                             \box\@outputbox \hss}%
%                                             }%
%                              }%
%    \@combinedblfloats
%    \@outputpage
%    \begingroup
%      \@dblfloatplacement
%      \@startdblcolumn
%      \@whilesw\if@fcolmade \fi
%        {\@outputpage
%         \@startdblcolumn}%
%    \endgroup
%  \fi
%}
%\makeatother

\DeclareMathOperator{\trace}{trace}
\DeclareMathOperator{\diag}{diag}