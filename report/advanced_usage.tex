\section{Advanced usage}
\label{sec:advanced_usage}

\subsection{Using the API}

\todo[inline=true, caption={}]{
\begin{itemize}
\item Higher order polynomials

\item More dyadic intervals. 

\item high accuracy approximations.


\item Parametrised distributions. 

\item Asymmetric distributions. 

\item Custom distributions.

\item For the custom distributions, showcase for a variety of these how fast they are. 

\item Discuss distributions which do not have an L2 norm, such as the Cauchy distribution, and how this related to the proofs in the ACM toms paper. 
\end{itemize}
}


\todo[inline=true]{Mention that if the polynomial order is known at compile time, then rather than a generic for loop, the polynomial evaluation can be unrolled, hence why we provide linear and cubic routines out of the box. Similarly for a fixed number of dyadic intervals we know the storage requirements for these and they can live on the stack rather than the heap, and sufficiently small intervals and low polynomial orders for non-parametric distributions can fit in the L1 cache or even in the registers, and can be read in a single cache line!}

