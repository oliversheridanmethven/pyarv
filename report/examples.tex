\section{Examples}
\label{sec:examples}

\todo[inline=true, caption={}]{
Give an example in the following settings:
\begin{itemize}
\item using a naive Rademacher Random variables and using a zeroth order single interval dyadic. 

\item Give a Gaussian example with simple Monte Carlo estimation (e.g. estiamte pi or something similar as an example of numerical integration unrelated to SDEs, or some high dimensional problem).

\item Give an SDE example pricing a call option, so we can compare to the analytic formula. 

\item Give a more involved CIR example with the non central chi-squared distribution. 

\item Give a multi-level Monte Carlo application showing the impact of the MLMC correction step, showing accuracy is maintained.

\item Find some example using a Possion distribution. 

\item Perhaps I can give an example for hypothesis testing parameters from some empirical distributions, such as parameter estimation for a Dirchlet distribution from a Bayesian framework. 


\item Perform a Bayesian Monte Carlo or Markov Chain Monte Carlo using some empirical distribution or some custom distribution which might be some polynomial (maybe a polynomial is a poor example, so perhaps a sinusoidal distribution with a discontinuity)

\item Can I give an example using a Cauchy distribution?

\item Find examples from outside Monte Carlo. (Chaos theory, weather simulation, random matrices, etc.)

\item Give an example using quasi-random numbers. 
\end{itemize}

For several of these examples, they could easily be measured by the metric of ``time to achieve a given accuracy'', or conversely ''accuracy achieved in a given time'' (e.g. 5-10 mins). I think the ``accuracy achieved in a given time'' for the Monte Carlo applications could be readily measured by using a timeout decorator (like the one in orsm) and then the error in the estimate (given by the empirical variance) can be compared. We can do a sanity check that the exact answer is within the expected bounds and see the exact level of error (although this is a random variable), so this might require several repetitions and some bootstrapping.  
}

